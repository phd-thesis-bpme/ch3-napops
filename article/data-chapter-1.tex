\chapter{Point count offsets for estimating population sizes of North American landbirds}

\section{Abstract}
Bird monitoring in North America over several decades has generated many open databases, housing millions of structured and semi-structured bird observations. These provide the opportunity to estimate bird densities and population sizes, once variation in factors such as underlying field methods, timing, land cover, proximity to roads, and uneven spatial coverage are accounted for. To facilitate that integration across databases, we introduce NA-POPS: Point Count Offsets for Population Sizes of North American Landbirds. NA-POPS is a large-scale, multi-agency project providing an open-source database of detectability functions for all North American landbirds. These detectability functions allow the integration of data from across disparate survey methods using the QPAD approach, which considers the probability of detection (q) and availability (p) of birds in relation to area (a) and density (d). To date, NA-POPS has compiled over 7.1 million data points spanning 292 projects from across North America, and produced detectability functions for 338 landbird species. Here, we describe the methods used to curate these data and generate these detectability functions, as well as the open-access nature of the resulting database. 

\section{Introduction}
\par The broad scale monitoring of birds in North America over the past several decades has resulted in the availability of millions of bird observations in open databases that span most of the continent. Individual programs such as the North American Breeding Bird Survey (BBS \citep{hudson_role_2017, sauer_first_2017}), the Integrated Monitoring in Bird Conservation Regions (IMBCR \citep{pavlacky_statistically_2017}), the Boreal Avian Modelling Project (BAM \citep{cumming_toward_2010}), and eBird \citep{sullivan_ebird_2014} provide a great deal of information on relative abundance over time and space. Partners in Flight (PIF) has previously estimated population sizes of landbirds using BBS data and a series of sophisticated expert-informed equations to extrapolate survey-level counts to total abundance within defined geographic regions \citep{rosenberg_setting_2005, will_handbook_2020}. These population sizes have been used to inform reports such as the 2019 State of Canada’s Birds report \citep{north_american_bird_conservation_initiative_canada_state_2019}, and to show the loss of nearly 3 billion North American birds since the 1970s \citep{rosenberg_decline_2019}.

\par An estimate of detectability is needed to translate survey-level counts into estimates of total abundance \citep{rosenberg_setting_2005, stanton_estimating_2019}. For a bird, the overall probability of detection (i.e., its detectability) can be broken down into two independent probabilities: availability and perceptibility \citep{marsh_correcting_1989, johnson_defense_2008}. Availability is defined as the probability of a bird giving a cue (auditory or visual) during a survey. This probability is a function of a bird’s cue rate, defined as the expected number of cues per unit time, and can be calculated using surveys that employ removal sampling \citep{barker_statistical_1995, farnsworth_removal_2002, alldredge_time--detection_2007}. Perceptibility is defined as the probability of an observer detecting a cue from a bird, provided the bird is actually giving a cue. This probability is a function of the bird’s effective detection radius (EDR), defined as the distance at which the same number of birds go detected and undetected, and can be calculated using data from surveys that employ distance sampling \citep{buckland_introduction_2001, buckland_distance_2015}.

\subsection{The QPAD Approach to Detectability Estimation}
\par Detectability in landbirds is generally non-constant. Factors such as time of day, time of year, habitat type, and presence of roads, among several other factors, have been shown to affect both the availability and the perceptibility of birds \citep{wilson_reliability_1985, solymos_calibrating_2013, johnston_species_2014, cooke_road_2020}. Additionally, the length of time for which an observer surveys for a bird, and the maximum survey distance for which the observer is surveying, can account for some variation in how many birds are detected and recorded for any given survey \citep{alldredge_factors_2007, solymos_calibrating_2013, buckland_distance_2015}.  

\par The QPAD method developed by BAM is a flexible approach to accounting for heterogeneity in survey conditions and survey methodology \citep{solymos_calibrating_2013}. It can calculate availability and perceptibility independently, while allowing for multiple surveying methods to be accounted for at once. In other words, any dataset that employs a removal sampling approach with two or more time bins can be jointly used to calculate availability, and any dataset that employs a distance sampling approach with two or more distance bins can be jointly used to calculate perceptibility (Sólymos et al. 2013). Additionally, by recognizing that availability is a function of cue rate, and that perceptibility is a function of EDR, the QPAD method allows for variation in cue rate and EDR as a function of covariates that affect detectability (such as time of day, time of year, habitat type, roadsides, etc.), and for estimates of perceptibility as a function of survey radius or other covariates (Sólymos et al. 2013). 

\par In general, at any given sampling event, for species $s$, there exists the true species abundance $Y_s$, which represents all the individuals of species $s$ at that site at the time of the sampling event. During any sampling event, the recorded count for species $s$ (i.e., $\hat{Y_s}$), is a subset of the true abundance $Y_s$, such that $\hat{Y_s} \leq Y_s$, assuming individuals are correctly identified \citep{bennett_how_nodate, johnson_defense_2008}. The difference between the recorded count and the true count depends on the detectability of the birds. The factors that affect detectability of species $s$ can be accounted for in the latent detectability function $f(\cdot)$, such that
\begin{equation}\label{detect-function}
	\mathbb{E}\left[\hat{Y_s}\right] = Y_s \times f(\cdot).
\end{equation}

\par Let $p_s(t)$ be the availability of species $s$ for a given survey duration $t$, and let $q_s(r)$ be the perceptibility of species $s$ for a given survey radius $r$ from an observer \citep{solymos_calibrating_2013}. The detectability function can now be written as $f(\cdot) = p_s(t) \times q_s(r)$, and Equation \ref{detect-function} can now be written as:

\begin{equation*}
	\mathbb{E}\left[\hat{Y_s}\right] = Y_s \times p_s(t) \times q_s(r).
\end{equation*}

\par $p_s$ is calculated from surveys that employ a removal sampling approach \citep{alldredge_time--detection_2007, farnsworth_removal_2002, solymos_evaluating_2018}, in which total survey time is partitioned into “time bins” $t_J$, not necessarily of equal size. For example, a 6-minute survey may be partitioned into $J = 6$ 1-minute time bins. Or, it could be partitioned into a 1-minute time bin, followed by a 3-minute time bin, followed by a 2-minute time bin, for a total of $J = 3$ time bands. If a bird gives a cue and is recorded in time bin $j$, then it is “removed” from the available birds to be recorded in all subsequent time bin. Cue rates follow a Poisson process with Poisson parameter $\phi$, which is estimated via multinomial conditional maximum likelihood estimation given survey type and any number of covariates, and the time to first cue event (i.e., detection) follows an exponential distribution $f(t) = \phi e^{-t\phi}$. Thus, the cumulative distribution function for time to first detection on the interval $\left[0, t_J\right]$ is given by (Alldredge et al., 2007)
\begin{align*}
	p(t_J) &= \int_{0}^{t_J} \phi e^{-t\phi} dt \\
	&= 1 - e^{-t\phi}.
\end{align*}

\par $q_s$ is calculated from surveys employing a distance sampling protocol \citep{buckland_introduction_2001, buckland_distance_2015}. Distance sampling partitions the point count survey area into discrete distance bins $r_K$. For example, when considering a 200m point count radius, the area may be partitioned into $K = 2$ 100m bins, or $K = 4$ 50m bins. The outermost bin can have finite or infinite truncation distance. Birds are then recorded by the observer as to how far away from the observer the bird was perceived. \citet{solymos_calibrating_2013} assume a half-normal detection function with probability of detecting an available bird at a distance $r$ from the observer is $g(r) = e^{-\dfrac{r^2}{\tau^2}}$, where $\tau^2$ is the variance of the unfolded normal distribution that describes the rate of distance decay and is estimated via multinomial conditional maximum likelihood estimation given survey type and any number of covariates. Following \citet{buckland_introduction_2001, solymos_calibrating_2013} the probability $q_s$ that an individual bird located within radius $r_K$ gives a cue that is perceived by an observer is given by
\begin{equation*}
	q(r_K) = \dfrac{\pi \tau^2 \left(1 - e^{-\dfrac{r^2}{\tau^2}}\right)}{\pi r^2_K}
\end{equation*}

\par Finally, the QPAD method allows for estimates of true density to be derived from any survey, by allowing the detectability function to act as a statistical offset to account for differences among survey types. An offset term is used in linear models to adjust the expected value with a known quantity. In our case the detectability function quantity is not known but estimated through QPAD. But as a result, the offsets allow all survey-observed counts to be translated into an estimate of true density. That is, by recognizing that $Y_s = D_s \times A$, where $D_s$ is the density of species $s$ and $A$ is the total area sampled at a point count, we can get that $\mathbb{E}\left[\hat{Y_s}\right] = D_s \times A \times p_s(t) \times q_s(r)$, which can be rearranged for density $D_s$. Note that through commutativity, we can rearrange these terms as $\mathbb{E}\left[\hat{Y_s}\right] = q_s(r) \times p_s(t) \times A \times D_s$, hence "QPAD" \citep{solymos_calibrating_2013}.

\subsection{The Need for a Systematic Estimate of Detectability}
\par The current PIF population size estimates use coarse binned estimates of detection distance for each landbird, and calculate uncertainty around the detection distance using a uniform distribution \citep{stanton_estimating_2019, will_handbook_2020}. However, the methods by which these binned estimates are determined are not consistent across species, and often rely only on expert opinion. There is therefore a need for a systematic approach to estimating these detection distances for all landbirds, while accounting for variation in environmental conditions and survey types \citep{stanton_estimating_2019, will_handbook_2020}. BAM has already made huge strides in accomplishing this, by first generating estimates of cue rate and EDR for 75 North American boreal birds \citep{solymos_calibrating_2013}, and then further extending that to cue rates of 151 boreal birds \citep{solymos_evaluating_2018}, each time using data harmonization techniques \citep{barker_ecological_2015} and the QPAD methodology \citep{solymos_calibrating_2013} to allow for multiple survey types and survey conditions to be accounted for. Additionally, QPAD offsets produced by BAM have been used extensively to adjust survey point count data to account for detectability \citep{hobson_long-term_2019, zlonis_burn_2019, knaggs_avian_2020, leston_quantifying_2020}, and to estimate population sizes and distribution of boreal birds \citep{crosby_differential_2019, solymos_lessons_2020}. Thus, the next frontier is to extend these methods developed by BAM and use the millions of rigorously collected bird observations and covariates (i.e., landcover and road networks) that are now available at a continental scale to derive detectability estimates for as many North American landbirds as possible.

We have therefore created the collaborative project NA-POPS: Point count Offsets for Population Sizes of North American Landbirds, to apply the QPAD approach developed by BAM to a compilation of point counts across North America. Our overarching goal is to generate an open-source database of detectability functions, thus creating a systematic and standardized approach to generating detectability estimates across North American landbird species. NA-POPS includes a GitHub organization \citep{blischak_quick_2016, crystal-ornelas_not_2022} to securely store the databases, a series of fitted models to estimate cue rates and EDRs in common observation conditions, and an R-package for users to access the estimates. Here, we detail the methods surrounding the following key components of achieving this large-scale project: 1) data acquisition and standardization, 2) derivation of covariates and modelling of cue rate and EDR, and 3) the software infrastructure used to curate the data, generate model runs, and host results. We summarize the results of the data collection and modelling efforts, and highlight and example of species-specific results of American Robin \textit{Turdus migratorius}, a suitable species for a case study as it is a wide-ranging North American landbird well-covered in the NA-POPS database. Finally, we discuss some applications for these detectability offsets, and invite further data contributions to enable additional refinement of the offsets produced by NA-POPS.

\section{Methods}
\subsection{Data Acquisition and Standardization}
We solicited point count data sets from across Canada and the United States that used either removal sampling, distance sampling, or both. Each data set was subject to data cleaning and standardization before being added to the NA-POPS database, following techniques initially developed for North America’s boreal region by BAM \citep{cumming_toward_2010, barker_ecological_2015}. For the purposes of this analysis, we considered one “sampling event” to be a single visit to a specific location to conduct a point-count survey. Some surveys were designed to include a transect or grid of point counts; in these cases, each of those point counts were considered unique sampling events. The following subsections detail each component of the standardization process.

\subsubsection{Non-species Removal}
We filtered and removed non-avian and non-species level data including records at the genus level or higher (e.g., hummingbird sp.). We used the 2020 American Ornithological Society checklist to update the data to address name changes, recently described species, and new splits and lumps \citep{chesser_sixty-first_2020}. Species that were previously lumped and now considered multiple species (e.g., Blue Grouse which is now Dusky Grouse and Sooty Grouse) were removed from the analysis. Species that were previously split and are now considered one species, or species that have recognized subspecies or forms were lumped into their parent species. 

\subsubsection{Unique Sample ID}
We generated a unique sample ID for each visit in a point count to identify a unique point count event (i.e., a survey from project P conducted at location x on day z). The unique sample ID is a string combination of the form \textless project\textgreater:\textless date/time\textgreater:\textless latitude\textgreater:\textless longitude\textgreater, where \textless project\textgreater is the name of the project from which the point count event comes from, \textless date/time\textgreater is the date and start time of the point count event in UTC (calculated using R packages \ttt{lutz}\citep{teucher_lutz_2019} and \ttt{proj4}\citep{urbanek_proj4_2020}), and \textless latitude\textgreater and \textless longitude\textgreater are latitude and longitude of the point count event. These sample IDs were used to cross-reference the individual point count in the observations data set to land cover and temporal covariates in their respective data sets.

\subsubsection{Time-of-Detection and Distance Binning}
TO DO: Need to create appendices for Table S1 and Table S2 from manuscript!!

Our analyses required that bird survey data were collected using either removal sampling, distance sampling, or both. As such, we classified each observation by their time-to-detection (TTD) and distance sampling methods using alphanumeric codes. Each unique TTD method was given an alphabetical code and each subinterval (or bin) within the method was assigned an increasing numeric value. For example, we assigned Method A to the protocol that contained the time subintervals of 0-2 mins (Level 1), 2-3 mins (Level 2), 3-4 mins (Level 3), …, 9-10 mins (Level 9; Supplemental Table S1). Distance sampling protocols were created in a similar fashion to TTD protocols. For example, we assigned Method A to the protocol that contained the distance subintervals of 0-50m (Level 1), 50-100m (Level 2), 100+m (Level 3; Supplemental Table S2).

While some data sets contained distance and/or time interval codes that could be renamed to match our NA-POPS codes, several data sets did not. For these datasets, we manually binned the data based on sampling protocol information within the metadata. Binning for distance sampling only required conditional expressions (e.g., “if distance < 50m, place in Level 1, otherwise place in Level 2”). In data sets that recorded exact distance, but did not explicitly bin the distance (such as the IMBCR data set), we opted to bin the distances based on Method CC, allowing for as much distance detail as possible to be preserved (Supplemental Table S2). 

Binning for TTD required some extra work. We could only manually bin TTD if the observation recorded the time at which the particular time bin started. That is, the “Time” column cannot simply be the start of the survey itself; it must be the start of the time bin. This check was conducted by reviewing the data set’s metadata to determine how the observations were recorded. Once that check was passed, the observations were grouped by their sample ID, described above. Then we checked to see that each sample ID (which corresponds to a single survey event) contains minimum and maximum times that are J intervals apart, according to that survey’s removal protocol. For example, if the protocol prescribed a 5-minute sample divided into 1-minute bins (i.e., J=5), then the minimum and maximum time for that survey event must differ by 5 minutes. If they differed by less than that amount, then we had a loss of information as to the true start time of the survey (because the minimum time may have been the 2nd time band if no birds were recorded in the 1st time band), and that survey event was dropped. This occurred on 7717 sampling events, all within the AKN data sets. Otherwise, if this check passed, then we manually generated bins based on the protocol and binned the observations.

\subsection{Modelling and Covariates}
\subsubsection{Removal Models}
We fitted 9 removal models \citep{farnsworth_removal_2002, alldredge_time--detection_2007} per species using the \ttt{detect} R package \citep{solymos_detect_2020} and different combinations of time-since-sunrise (TSSR), Ordinal day (OD), and their quadratic terms to account for possible unimodal relationships:

\begin{flalign*}
	\log Y_S | D &= \beta_0 \tag*{Model 1} \\
	\log Y_S | D &= \beta_0 + \left(\beta_1 \times OD\right) \tag*{Model 2} \\
	\log Y_S | D &= \beta_0 + \left(\beta_1 \times OD\right) + \left(\beta_2 \times OD^2\right)\tag*{Model 3} \\
	\log Y_S | D &= \beta_0 + \left(\beta_1 \times TSSR\right)\tag*{Model 4} \\
	\log Y_S | D &= \beta_0 + \left(\beta_1 \times TSSR\right) + \left(\beta_2 \times TSSR^2\right)\tag*{Model 5} \\
	\log Y_S | D &= \beta_0 + \left(\beta_1 \times TSSR\right) + \left(\beta_2 \times OD\right)\tag*{Model 6} \\
	\log Y_S | D &= \beta_0 + \left(\beta_1 \times TSSR\right) + \left(\beta_2 \times OD\right) + \left(\beta_3 \times OD^2\right)\tag*{Model 7} \\
	\log Y_S | D &= \beta_0 + \left(\beta_1 \times TSSR\right) + \left(\beta_2 \times TSSR^2\right) + \left(\beta_3 \times OD\right)\tag*{Model 8} \\
	\log Y_S | D &= \beta_0 + \left(\beta_1 \times TSSR\right) + \left(\beta_2 \times TSSR^2\right) + \left(\beta_3 \times OD\right) + \left(\beta_3 \times OD^2\right)\tag*{Model 9} \\
\end{flalign*}

\begin{figure}[h]
	\centering
	\includegraphics[width=\linewidth]{coherence_k500}
	\caption{\label{fig:coherence}This caption explains the above plot.}
\end{figure}

Looking at the figure above (\autoref{fig:coherence}), I don't remember what I was going to say.





\section{Third section}

Everything I do is in R 4.0\citep{R-4.0.0}. \ttt{purrr}\citep{purrr} is a great package. I also used a
bunch of other packages\citep{qs,dplyr,ggplot2}.
