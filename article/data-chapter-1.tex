\chapter{Point count offsets for estimating population sizes of North American landbirds}

%\section{Abstract}
%Bird monitoring in North America over several decades has generated many open databases, housing millions of structured and semi-structured bird observations. These provide the opportunity to estimate bird densities and population sizes, once variation in factors such as underlying field methods, timing, land cover, proximity to roads, and uneven spatial coverage are accounted for. To facilitate that integration across databases, we introduce NA-POPS: Point Count Offsets for Population Sizes of North American Landbirds. NA-POPS is a large-scale, multi-agency project providing an open-source database of detectability functions for all North American landbirds. These detectability functions allow the integration of data from across disparate survey methods using the QPAD approach, which considers the probability of detection (q) and availability (p) of birds in relation to area (a) and density (d). To date, NA-POPS has compiled over 7.1 million data points spanning 292 projects from across North America, and produced detectability functions for 338 landbird species. Here, we describe the methods used to curate these data and generate these detectability functions, as well as the open-access nature of the resulting database. 

\section{Introduction}
\par The broad scale monitoring of birds in North America over the past several decades has resulted in the availability of millions of bird observations in open databases that span most of the continent. Individual programs such as the North American Breeding Bird Survey (BBS \citep{hudson_role_2017, sauer_first_2017}), the Integrated Monitoring in Bird Conservation Regions (IMBCR \citep{pavlacky_statistically_2017}), the Boreal Avian Modelling Project (BAM \citep{cumming_toward_2010}), and eBird \citep{sullivan_ebird_2014} provide a great deal of information on relative abundance over time and space. Partners in Flight (PIF) has previously estimated population sizes of landbirds using BBS data and a series of sophisticated expert-informed equations to extrapolate survey-level counts to total abundance within defined geographic regions \citep{rosenberg_setting_2005, will_handbook_2020}. These population sizes have been used to inform reports such as the 2019 State of Canada’s Birds report \citep{north_american_bird_conservation_initiative_canada_state_2019}, and to show the loss of nearly 3 billion North American birds since the 1970s \citep{rosenberg_decline_2019}.

\par An estimate of detectability is needed to translate survey-level counts into estimates of total abundance \citep{rosenberg_setting_2005, stanton_estimating_2019}. For a bird, the overall probability of detection (i.e., its detectability) can be broken down into two independent probabilities: availability and perceptibility \citep{marsh_correcting_1989, johnson_defense_2008}. Availability is defined as the probability of a bird giving a cue (auditory or visual) during a survey. This probability is a function of a bird’s cue rate, defined as the expected number of cues per unit time, and can be calculated using surveys that employ removal sampling \citep{barker_statistical_1995, farnsworth_removal_2002, alldredge_time--detection_2007}. Perceptibility is defined as the probability of an observer detecting a cue from a bird, provided the bird is actually giving a cue. This probability is a function of the bird’s effective detection radius (EDR), defined as the distance at which the same number of birds go detected and undetected, and can be calculated using data from surveys that employ distance sampling \citep{buckland_introduction_2001, buckland_distance_2015}.

%\subsection{The QPAD Approach to Detectability Estimation}
%\par Detectability in landbirds is generally non-constant. Factors such as time of day, time of year, habitat type, and presence of roads, among several other factors, have been shown to affect both the availability and the perceptibility of birds \citep{wilson_reliability_1985, solymos_calibrating_2013, johnston_species_2014, cooke_road_2020}. Additionally, the length of time for which an observer surveys for a bird, and the maximum survey distance for which the observer is surveying, can account for some variation in how many birds are detected and recorded for any given survey \citep{alldredge_factors_2007, solymos_calibrating_2013, buckland_distance_2015}.  

%\par The QPAD method developed by BAM is a flexible approach to accounting for heterogeneity in survey conditions and survey methodology \citep{solymos_calibrating_2013}. It can calculate availability and perceptibility independently, while allowing for multiple surveying methods to be accounted for at once. In other words, any dataset that employs a removal sampling approach with two or more time bins can be jointly used to calculate availability, and any dataset that employs a distance sampling approach with two or more distance bins can be jointly used to calculate perceptibility (Sólymos et al. 2013). Additionally, by recognizing that availability is a function of cue rate, and that perceptibility is a function of EDR, the QPAD method allows for variation in cue rate and EDR as a function of covariates that affect detectability (such as time of day, time of year, habitat type, roadsides, etc.), and for estimates of perceptibility as a function of survey radius or other covariates (Sólymos et al. 2013). 

%\par In general, at any given sampling event, for species $s$, there exists the true species abundance $Y_s$, which represents all the individuals of species $s$ at that site at the time of the sampling event. During any sampling event, the recorded count for species $s$ (i.e., $\hat{Y_s}$), is a subset of the true abundance $Y_s$, such that $\hat{Y_s} \leq Y_s$, assuming individuals are correctly identified \citep{bennett_how_nodate, johnson_defense_2008}. The difference between the recorded count and the true count depends on the detectability of the birds. The factors that affect detectability of species $s$ can be accounted for in the latent detectability function $f(\cdot)$, such that
%\begin{equation}\label{detect-function}
%	\mathbb{E}\left[\hat{Y_s}\right] = Y_s \times f(\cdot).
%\end{equation}

%\par Let $p_s(t)$ be the availability of species $s$ for a given survey duration $t$, and let $q_s(r)$ be the perceptibility of species $s$ for a given survey radius $r$ from an observer \citep{solymos_calibrating_2013}. The detectability function can now be written as $f(\cdot) = p_s(t) \times q_s(r)$, and Equation \ref{detect-function} can now be written as:

%\begin{equation*}
%	\mathbb{E}\left[\hat{Y_s}\right] = Y_s \times p_s(t) \times q_s(r).
%\end{equation*}

%\par $p_s$ is calculated from surveys that employ a removal sampling approach \citep{alldredge_time--detection_2007, farnsworth_removal_2002, solymos_evaluating_2018}, in which total survey time is partitioned into “time bins” $t_J$, not necessarily of equal size. For example, a 6-minute survey may be partitioned into $J = 6$ 1-minute time bins. Or, it could be partitioned into a 1-minute time bin, followed by a 3-minute time bin, followed by a 2-minute time bin, for a total of $J = 3$ time bands. If a bird gives a cue and is recorded in time bin $j$, then it is “removed” from the available birds to be recorded in all subsequent time bin. Cue rates follow a Poisson process with Poisson parameter $\phi$, which is estimated via multinomial conditional maximum likelihood estimation given survey type and any number of covariates, and the time to first cue event (i.e., detection) follows an exponential distribution $f(t) = \phi e^{-t\phi}$. Thus, the cumulative distribution function for time to first detection on the interval $\left[0, t_J\right]$ is given by (Alldredge et al., 2007)
%\begin{align*}
%	p(t_J) &= \int_{0}^{t_J} \phi e^{-t\phi} dt \\
%	&= 1 - e^{-t\phi}.
%\end{align*}

%\par $q_s$ is calculated from surveys employing a distance sampling protocol \citep{buckland_introduction_2001, buckland_distance_2015}. Distance sampling partitions the point count survey area into discrete distance bins $r_K$. For example, when considering a 200m point count radius, the area may be partitioned into $K = 2$ 100m bins, or $K = 4$ 50m bins. The outermost bin can have finite or infinite truncation distance. Birds are then recorded by the observer as to how far away from the observer the bird was perceived. \citet{solymos_calibrating_2013} assume a half-normal detection function with probability of detecting an available bird at a distance $r$ from the observer is $g(r) = e^{-\dfrac{r^2}{\tau^2}}$, where $\tau^2$ is the variance of the unfolded normal distribution that describes the rate of distance decay and is estimated via multinomial conditional maximum likelihood estimation given survey type and any number of covariates. Following \citet{buckland_introduction_2001, solymos_calibrating_2013} the probability $q_s$ that an individual bird located within radius $r_K$ gives a cue that is perceived by an observer is given by
%\begin{equation*}
%	q(r_K) = \dfrac{\pi \tau^2 \left(1 - e^{-\dfrac{r^2}{\tau^2}}\right)}{\pi r^2_K}
%\end{equation*}

%\par Finally, the QPAD method allows for estimates of true density to be derived from any survey, by allowing the detectability function to act as a statistical offset to account for differences among survey types. An offset term is used in linear models to adjust the expected value with a known quantity. In our case the detectability function quantity is not known but estimated through QPAD. But as a result, the offsets allow all survey-observed counts to be translated into an estimate of true density. That is, by recognizing that $Y_s = D_s \times A$, where $D_s$ is the density of species $s$ and $A$ is the total area sampled at a point count, we can get that $\mathbb{E}\left[\hat{Y_s}\right] = D_s \times A \times p_s(t) \times q_s(r)$, which can be rearranged for density $D_s$. Note that through commutativity, we can rearrange these terms as $\mathbb{E}\left[\hat{Y_s}\right] = q_s(r) \times p_s(t) \times A \times D_s$, hence "QPAD" \citep{solymos_calibrating_2013}.

\subsection{The Need for a Systematic Estimate of Detectability}
\par The current PIF population size estimates use coarse binned estimates of detection distance for each landbird, and calculate uncertainty around the detection distance using a uniform distribution \citep{stanton_estimating_2019, will_handbook_2020}. However, the methods by which these binned estimates are determined are not consistent across species, and often rely only on expert opinion. There is therefore a need for a systematic approach to estimating these detection distances for all landbirds, while accounting for variation in environmental conditions and survey types \citep{stanton_estimating_2019, will_handbook_2020}. BAM has already made huge strides in accomplishing this, by first generating estimates of cue rate and EDR for 75 North American boreal birds \citep{solymos_calibrating_2013}, and then further extending that to cue rates of 151 boreal birds \citep{solymos_evaluating_2018}, each time using data harmonization techniques \citep{barker_ecological_2015} and the QPAD methodology \citep{solymos_calibrating_2013} to allow for multiple survey types and survey conditions to be accounted for. Additionally, QPAD offsets produced by BAM have been used extensively to adjust survey point count data to account for detectability \citep{hobson_long-term_2019, zlonis_burn_2019, knaggs_avian_2020, leston_quantifying_2020}, and to estimate population sizes and distribution of boreal birds \citep{crosby_differential_2019, solymos_lessons_2020}. Thus, the next frontier is to extend these methods developed by BAM and use the millions of rigorously collected bird observations and covariates (i.e., landcover and road networks) that are now available at a continental scale to derive detectability estimates for as many North American landbirds as possible.

\par We have therefore created the collaborative project NA-POPS: Point count Offsets for Population Sizes of North American Landbirds, to apply the QPAD approach developed by BAM to a compilation of point counts across North America. Our overarching goal is to generate an open-source database of detectability functions, thus creating a systematic and standardized approach to generating detectability estimates across North American landbird species. NA-POPS includes a GitHub organization \citep{blischak_quick_2016, crystal-ornelas_not_2022} to securely store the databases, a series of fitted models to estimate cue rates and EDRs in common observation conditions, and an R-package for users to access the estimates. Here, we detail the methods surrounding the following key components of achieving this large-scale project: 1) data acquisition and standardization, 2) derivation of covariates and modelling of cue rate and EDR, and 3) the software infrastructure used to curate the data, generate model runs, and host results. We summarize the results of the data collection and modelling efforts, and highlight and example of species-specific results of American Robin \textit{Turdus migratorius}, a suitable species for a case study as it is a wide-ranging North American landbird well-coveredw in the NA-POPS database. Finally, we discuss some applications for these detectability offsets, and invite further data contributions to enable additional refinement of the offsets produced by NA-POPS.


\section{Methods}
\subsection{Data Acquisition and Standardization}
\par We solicited point count data sets from across Canada and the United States that used either removal sampling, distance sampling, or both. Each data set was subject to data cleaning and standardization before being added to the NA-POPS database, following techniques initially developed for North America’s boreal region by BAM \citep{cumming_toward_2010, barker_ecological_2015}. For the purposes of this analysis, we considered one “sampling event” to be a single visit to a specific location to conduct a point-count survey. Some surveys were designed to include a transect or grid of point counts; in these cases, each of those point counts were considered unique sampling events. The following subsections detail each component of the standardization process.

\subsubsection{Non-species Removal}
\par We filtered and removed non-avian and non-species level data including records at the genus level or higher (e.g., hummingbird sp.). We used the 2020 American Ornithological Society checklist to update the data to address name changes, recently described species, and new splits and lumps \citep{chesser_sixty-first_2020}. Species that were previously lumped and now considered multiple species (e.g., Blue Grouse which is now Dusky Grouse and Sooty Grouse) were removed from the analysis. Species that were previously split and are now considered one species, or species that have recognized subspecies or forms were lumped into their parent species. 

\subsubsection{Unique Sample ID}
\par We generated a unique sample ID for each visit in a point count to identify a unique point count event (i.e., a survey from project P conducted at location x on day z). The unique sample ID is a string combination of the form:

 $$\textless project\textgreater:\textless date/time\textgreater:\textless latitude\textgreater:\textless longitude\textgreater$$
 where \textless project\textgreater  is the name of the project from which the point count event comes from, \textless date/time\textgreater  is the date and start time of the point count event in UTC (calculated using R packages \ttt{lutz}\citep{teucher_lutz_2019} and \ttt{proj4}\citep{urbanek_proj4_2020}), and \textless latitude\textgreater  and \textless longitude\textgreater  are latitude and longitude of the point count event. These sample IDs were used to cross-reference the individual point count in the observations data set to land cover and temporal covariates in their respective data sets.

\subsubsection{Time-of-Detection and Distance Binning}
\par TO DO: Need to create appendices for Table S1 and Table S2 from manuscript!!

\par Our analyses required that bird survey data were collected using either removal sampling, distance sampling, or both. As such, we classified each observation by their time-to-detection (TTD) and distance sampling methods using alphanumeric codes. Each unique TTD method was given an alphabetical code and each subinterval (or bin) within the method was assigned an increasing numeric value. For example, we assigned Method A to the protocol that contained the time subintervals of 0-2 mins (Level 1), 2-3 mins (Level 2), 3-4 mins (Level 3), …, 9-10 mins (Level 9; Supplemental Table S1). Distance sampling protocols were created in a similar fashion to TTD protocols. For example, we assigned Method A to the protocol that contained the distance subintervals of 0-50m (Level 1), 50-100m (Level 2), 100+m (Level 3; Supplemental Table S2).

\par While some data sets contained distance and/or time interval codes that could be renamed to match our NA-POPS codes, several data sets did not. For these datasets, we manually binned the data based on sampling protocol information within the metadata. Binning for distance sampling only required conditional expressions (e.g., “if distance < 50m, place in Level 1, otherwise place in Level 2”). In data sets that recorded exact distance, but did not explicitly bin the distance (such as the IMBCR data set), we opted to bin the distances based on Method CC, allowing for as much distance detail as possible to be preserved (Supplemental Table S2). 

\par Binning for TTD required some extra work. We could only manually bin TTD if the observation recorded the time at which the particular time bin started. That is, the “Time” column cannot simply be the start of the survey itself; it must be the start of the time bin. This check was conducted by reviewing the data set’s metadata to determine how the observations were recorded. Once that check was passed, the observations were grouped by their sample ID, described above. Then we checked to see that each sample ID (which corresponds to a single survey event) contains minimum and maximum times that are J intervals apart, according to that survey’s removal protocol. For example, if the protocol prescribed a 5-minute sample divided into 1-minute bins (i.e., J=5), then the minimum and maximum time for that survey event must differ by 5 minutes. If they differed by less than that amount, then we had a loss of information as to the true start time of the survey (because the minimum time may have been the 2nd time band if no birds were recorded in the 1st time band), and that survey event was dropped. This occurred on 7717 sampling events, all within the AKN data sets. Otherwise, if this check passed, then we manually generated bins based on the protocol and binned the observations.

\subsection{Modelling and Covariates}
\subsubsection{Removal Models}
\par We fitted 9 removal models \citep{farnsworth_removal_2002, alldredge_time--detection_2007} per species using the \ttt{detect} R package \citep{solymos_detect_2020} by considering different combinations of time-since-sunrise (TSSR), Ordinal day (OD), and their quadratic terms to account for possible unimodal relationships:

\begin{flalign*}
	\log Y_S | D &= \beta_0 \tag*{Model 1} \\
	\log Y_S | D &= \beta_0 + \left(\beta_1 \times OD\right) \tag*{Model 2} \\
	\log Y_S | D &= \beta_0 + \left(\beta_1 \times OD\right) + \left(\beta_2 \times OD^2\right)\tag*{Model 3} \\
	\log Y_S | D &= \beta_0 + \left(\beta_1 \times TSSR\right)\tag*{Model 4} \\
	\log Y_S | D &= \beta_0 + \left(\beta_1 \times TSSR\right) + \left(\beta_2 \times TSSR^2\right)\tag*{Model 5} \\
	\log Y_S | D &= \beta_0 + \left(\beta_1 \times TSSR\right) + \left(\beta_2 \times OD\right)\tag*{Model 6} \\
	\log Y_S | D &= \beta_0 + \left(\beta_1 \times TSSR\right) + \left(\beta_2 \times OD\right) + \left(\beta_3 \times OD^2\right)\tag*{Model 7} \\
	\log Y_S | D &= \beta_0 + \left(\beta_1 \times TSSR\right) + \left(\beta_2 \times TSSR^2\right) + \left(\beta_3 \times OD\right)\tag*{Model 8} \\
	\log Y_S | D &= \beta_0 + \left(\beta_1 \times TSSR\right) + \left(\beta_2 \times TSSR^2\right) + \left(\beta_3 \times OD\right) + \left(\beta_3 \times OD^2\right)\tag*{Model 9} \\
\end{flalign*}

\par In each model, $Y_s$ is the observed abundance for species $s$, $D$ is the design matrix for the removal survey protocols (see Table TO DO), and the vector of $\beta$s are the associated effects of each covariate. For all models, we analyzed only the subset of data that contained two or more subintervals of time, and for species for which we had 75 or more sampling events that contained at least one detection \citep{matsuoka_using_2012, solymos_evaluating_2018}. We ranked candidate models using Akaike’s information criterion (AIC) to select the best supported model (i.e., lowest AIC score) for each species \citep{akaike_new_1974}.

\par TSSR was calculated in R using the “maptools” package, which has functionality to calculate the sunrise time for a location given a date \citep{bivand_maptools_2020}. Only data that included locational information (latitude and longitude) and start time and date were able to have TSSR calculated, otherwise the data had to be filtered out. For each species, we centred each TSSR value prior to modelling by the species-specific median TSSR, and divided all values by their maximum possible value of 24.

\par OD was calculated by converting the standardized UTC time into the day of the year. For each species, we centred each OD value prior to modelling by the species-specific median OD, and divided all OD values by 365. 

\subsubsection{Distance Models}
\par We fitted 5 distance models per species using using the \ttt{detect} R package \citep{solymos_detect_2020} by considering different combinations of roadside status (ROAD) and forest coverage (FC), to account for differences in sound attenuation and visibility in these different environments \citep{yip_sound_2017}:

\begin{flalign*}
	\log Y_S | D &= \beta_0 \tag*{Model 1} \\
	\log Y_S | D &= \beta_0 + \left(\beta_1 \times ROAD\right) \tag*{Model 2} \\
	\log Y_S | D &= \beta_0 + \left(\beta_1 \times FC\right)\tag*{Model 3} \\
	\log Y_S | D &= \beta_0 + \left(\beta_1 \times ROAD\right) + \left(\beta_2 \times FC\right)\tag*{Model 4} \\
	\log Y_S | D &= \beta_0 + \left(\beta_1 \times ROAD\right) + \left(\beta_2 \times FC\right) + \left(\beta_3 \times \right[ROAD \times FC \left]\right)\tag*{Model 5} \\
\end{flalign*}

\par In each model, $Y_s$ is the observed abundance for species $s$, $D$ is the design matrix for the distance survey protocols (see Table TO DO), and the vector of $\beta$s are the associated effects of each covariate. We analyzed only the subset of data that contained 2 or more subintervals of distance, and for species for which we had 75 or more sampling events that contained at least one detection \citep{buckland_distance_2015, matsuoka_using_2012}. For each point count location in the database, two spatial covariates were calculated: the distance to the nearest road and land cover type. Only data that included locational information (latitude and longitude) were able to have these covariates calculated, otherwise the data had to be filtered out. As for removal models, the best supported model for each species was evaluated using AIC.

\par Road data from Statistics Canada \citep{statistics_canada_intercensal_2019}, the United States Census Bureau \citep{us_geological_survey_usgs_2020}, and the Mexican National Institute of Statistics, Geography and Informatics \citep{national_institute_of_statistics_geography_and_informatics_communication_2019} were assembled, reprojected, and clipped to retain only data within 10km of each point count location. For each point count location, the distance to the nearest road was calculated using the “Near” tool in ArcGIS 10.7 \citep{environmental_systems_research_institute_arcgis_2011}. 

\par The 2015 North American Land Change Monitoring System (NALCMS) provided a standardized and seamless landcover dataset for the entire study area \citep{natural_resources_canada_2010-2015_2020}. The classification includes 19 landcover classes defined using the Level II Land Cover Classification System (LCCS) standard developed by the Food and Agriculture Organization (FAO) of the United Nations. The 19 cover classes were collapsed into two classes, forested vs. non-forested. We then calculated the proportion of forested area (hereinafter, forest coverage) surrounding each point count location at a 150-m resolution (5×5 pixel analysis).

\subsection{NA-POPS Infrastructure}
\par We used GitHub Organizations \citep{braga_not_2023} to organize both raw data from individual projects, and for scripts related to combining these data and generating the detectability functions. This allowed for the results to be open access in a central repository, and for all original data from data providers to be private.

\par \autoref{fig:github} provides a visual overview of the GitHub infrastructure for NA-POPS. Project-level data sets were hosted in their own repository that takes on the naming convention “project-NAME”, where NAME is replaced by the project name. Some projects, such as the Boreal Avian Modelling project, are a collection of smaller individual projects that have their own sampling methods; such projects were considered “metaprojects” and took on the naming convention “project-NAME\_m”, where the “\_m” denotes the metaproject. Each of these repositories were organized as follows: metadata and citation information were stored in the top level of the project’s repository; raw data files directly from the data source were stored in the “rawdata” subdirectory; scripts to manipulate the raw data into a common data form (as described in Data Standardization) were stored in the “src” subdirectory; the standardized data files were stored in the “output” subdirectory. Hosting each data set in its own repository allowed for ease of tracking scripts needed to standardize each data set, and further allowed ease of computation when combining these data into the QPAD calculations. By default, each data repository was set to “private” so that only those with permission (i.e., organization administrators, or data owners with Github accounts) may access the repository. 

\par A “covariates” repository contained the covariate data sets (as subdirectories), as well as any scripts needed to generate the covariates, used in the removal and distance modeling. The covariates repository also stored the survey protocols for both the removal and distance surveys.

\par An “analysis” repository contained the scripts needed to a) combine all individual standardized data sets, b) generate both the count matrices (though this repository did not host the combined data) and design matrices (Tables TO DO and TO DO) using the R package “reshape2” \citep{wickham_reshaping_2007}, and c) run the models to estimate coefficients for calculating $\phi$ and $\tau$. 

\par The parameter estimates extracted and saved from the modelling step of the “analysis” repository, as well as the variance-covariance matrix for each species-model combination, were stored in a public “results” repository.
Throughout each repository, RStudio Projects were used to allow scripts to access files in other repositories using relative paths.

\begin{figure}[h]
	\centering
	\includegraphics[width=\linewidth]{../ch3-napops/output/github.png}
	\caption{\label{fig:github}General structure of the NA-POPS Github Organization.  Raw data (a) in the form of metaprojects (black boxes) or individual projects (blue boxes) are made into their own Github (b) project repository (red boxes), where the raw data is run through a script to standardize the data into a common format. These standardized data sets, along with the landcover and temporal covariates, are combined in the “analysis” repository where the data are modelled using the removal and distance sampling models. The coefficients from these are calculated and output into a public “results” repository.}
\end{figure}

\section{Results}

The full suite of results is visualized on the NA-POPS dashboard at https://na-pops.org/. Additionally, researchers can begin to explore and apply these offsets by using the R package \ttt{napops} \citep{edwards_napops_2024}. This R package includes a README file that demonstrates how to use access estimates of cue rate and EDR through the package, as well as estimating probability of availability and probability of perceptibility. All post-hoc analyses, including generation of figures and tables, were performed with this R package. 

\subsection{Data Collection}
\par The NA-POPS Github organization can be found at https://github.com/na-pops. At the time of this paper, the NA-POPS database contains data from 292 individual projects (listed in Supplemental Table S3). These projects contributed a total of 7,144,709 landbird observations across 712,138 sampling events, 422,514 of which had sufficient ancillary data for removal modelling, and 522,820 of which had sufficient ancillary data for distance modelling. These sampling events contributed enough data to derive estimates of cue rate or effective detection radius for 338 species of North American landbirds, 319 of which had sufficient data for both (Supplemental Table S4). The sampling events represent a wide geographical range across Canada and the United States, including data from all but two (BCR 20: Edwards Plateau, and BCR 36: Tamaulipan Brushlands) of the 37 BCRs in Canada and the United States (\autoref{fig:ch1-map}a). In general, areas with a greater numbers of sampling events corresponded to areas where there were a greater number of projects contributing data. Some exceptions to this were the montane regions of the US, and the Great Lakes region, where a small number of projects contributed the majority of the data (\autoref{fig:ch1-map}b).

\begin{figure}[h]
	\centering
	\includegraphics[width=\linewidth]{../ch3-napops/output/plots/Fig1-coverage-map.png}
	\caption{\label{fig:ch1-map}Spatial coverage map of all sampling events considered in the analysis of this paper (A), and number of individual projects that contributed data to each stratum (B), stratified by Bird Conservation Region (BCR). Grey regions indicate no data (BCR 20: Edwards Plateau and BCR 36: Tamaulipan Brushlands).}
\end{figure}

\par We were able to compile observations that span a wide range of sampling covariates (\autoref{fig:ch1-covariates}). For removal modelling, OD covariates ranged from 61 to 244, with a median of 160; TSSR covariates ranged from –3.00 to 14.4, with a median of 1.64. Of those samples used for removal modelling, 43.6\% had a maximum survey duration of 10 minutes and 34.1\% had a maximum survey duration of 6 minutes. The remaining 22.3\% of removal samples consisted of maximum survey durations of 3 minutes, 5 minutes, or 8 minutes. For distance modelling, forest coverage covariates ranged from 0 – 1, with most sampling events having a value of either 0 (i.e., open canopy) or 1 (i.e., closed canopy); there was a bias toward offroad surveys (n = 414,555) compared to on-road surveys (n = 108,265). Of those samples used for distance modelling, 88.1\% used infinite radius point counts and 11.2\% used a maximum radius of 400 m. The remaining 0.70\% of the distance samples had a maximum survey radius of 30 m, 75 m, 100 m, or 150 m.

\begin{figure}[h]
	\centering
	\includegraphics[width=\linewidth]{../ch3-napops/output/plots/Fig2-covariates.png}
	\caption{\label{fig:ch1-covariates}Covariate space for all covariates considered for both removal modelling and distance modelling. (A) shows a 2D density plot for all values of Ordinal Day (OD) and Time Since Sunrise (TSSR) collected by NA-POPS, (B) shows a bar chart for the number of surveys containing each maximum survey time, (C) shows a histogram for Forest Coverage values colour-coded by roadside status, and (D) shows a bar chart for the number of surveys containing each maximum survey radius.}
\end{figure}

\subsection{Model Selection}
\par For the 319 species that had sufficient data for both removal modelling and distance modelling, the best model included at least one of the removal or distance covariates for all but 7 species (\autoref{fig:ch1-heatmap}). Two-hundred and eighty species (87.8\%) had a removal model selected that included at least one covariate; of these, 237 species included an OD term, 147 of which included the quadratic OD term; and 215 included a TSSR term, 108 of which included the quadratic TSSR term. Fifty-one out of the 319 species had the full model (Model 9) selected. Two-hundred and ninety-six species (92.8\%) had a distance model selected that included at least one covariate; of these, 261 species included a roadside status term, and 269 species included a forest coverage term. One-hundred and seventy-one out of the 319 species had the full model (Model 5) selected.

\begin{figure}[h]
	\centering
	\includegraphics[width=\linewidth]{../ch3-napops/output/plots/Fig3-model-selection.png}
	\caption{\label{fig:ch1-heatmap}Heatmap of model selection (chosen by AIC) for species that had sufficient data for both removal and distance modelling. Numbers inside squares indicate the number of species that had that particular removal model/distance model combination selected. Numbers in the margin are total number of species for that particular removal model or distance model.}
\end{figure}

\par Species with more sampling events tended to have more complex models chosen (Figure 4). For removal modelling, the mean sample size of species with null models selected was 1650, with a range between 87 (Florida Scrub-Jay Aphelocoma coerulescens) and 10120 (MacGillivray’s Warbler Geothlypis tolmiei). For distance modelling, the mean sample size of species with null models selected was 1612, with a range between 121 (Ferruginous Hawk) and 19760 (Vesper Sparrow).

\begin{figure}[h]
	\centering
	\includegraphics[width=\linewidth]{../ch3-napops/output/plots/Fig4-model-complexity.png}
	\caption{\label{fig:ch1-complexity}Species sample size (i.e., square root of the number of sampling events) grouped by complexity of the best model (based on number of covariates) as determined by AIC, for both removal (purple) and distance (yellow) modelling. See Supplemental Material for the list of models.}
\end{figure}

\subsection{The effects of Ordinal Day and Time Since Sunrise on availability}
\par We had sufficient data to analyze 332 species of landbirds across 46 families using removal models that contained OD, TSSR, and/or their quadratic terms as covariates (Supplemental Table S5). Note that this total includes species that may not have had sufficient sample size for distance modelling. \autoref{fig:ch1-removal-family} shows predicted availability curves for species in the top 4 families by sample size modelled by NA-POPS, plotted against varying values of OD and TSSR, for surveys of 5 minutes in duration. For most of these species, availability peaked around the 160th – 180th ordinal day (9 June – 29 June in a non-leap year) when keeping TSSR constant at its median of 1.6, and tended to decrease as time since sunrise increased, with some species showing some slight peaks between 0 – 2 hours after sunrise. 

\begin{figure}[h]
	\centering
	\includegraphics[width=\linewidth]{../ch3-napops/output/plots/Fig5-removal-family.png}
	\caption{\label{fig:ch1-removal-family}Plots of availability vs Ordinal Day (left) and Time Since Sunrise (right) for the four top families (by sample size) modelled in NA-POPS. Grey lines are species-specific availability curves within a family, and black lines are family-specific mean availability curves.}
\end{figure}

\subsection{The effects of Roadside Status and Forest Coverage on perceptibility}
\par We had sufficient data to analyze 325 species of landbirds across 45 families using distance models that contained Roadside Status, Forest Coverage, and their interaction as covariates (Supplemental Table S6). Note that this total includes species that may not have had sufficient sample size for removal modelling. \autoref{fig:ch1-distance-family} shows predicted effective detection radii for species in the top 4 families by sample size modelled by NA-POPS, plotted against varying values of Forest Coverage, for roadside and offroad survey. In both roadside and offroad surveys, effective detection radius, on average, decreased as forest coverage increased, with variability in magnitude of decrease among species within each family. 

\begin{figure}[h]
	\centering
	\includegraphics[width=\linewidth]{../ch3-napops/output/plots/Fig6-distance-family.png}
	\caption{\label{fig:ch1-distance-family}Plot of effective detection radius vs forest coverage for roadside and offroad surveys for the four top families (by sample size) modelled in NA-POPS. Grey lines are species-specific effective detection radii within a family, and black lines are family-specific mean effective detection radius curves.}
\end{figure}

\par For most species, roadside EDRs are greater than off-road EDRs (i.e., detectability is greater on roadsides than off-road) when forest coverage is high and the opposite is true (detectability is greater off-road than on roadsides) when forest coverage is low (\autoref{fig:ch1-edr-change}). The effects of roadside vs. offroad surveys and their interaction with forest coverage can be seen in Figure 7, where the change in EDR going from a roadside survey to an offroad survey (i.e., $\Delta EDR = EDR_{Roadside} - EDR_{Offroad}$) is plotted against increasing forest coverage; that is, positive values of $\Delta EDR$ mean that the roadside EDR is greater than the offroad EDR, and negative values of $\Delta EDR$ mean that the roadside EDR is less than the offroad EDR. For the 4 families considered here, there is a small increase in EDR moving from roadside to offroad surveys when forest coverage is lower (< 0.50), and then a slight decrease in EDR moving from roadside to offroad surveys when forest coverage is higher (> 0.50). The exception appears to be the Picidae family, where the change in EDR is negative throughout the values of forest coverage, indicating that EDR increases from roadside to offroad surveys no matter the forest coverage.

\begin{figure}[hbtp]
	\centering
	\includegraphics[width=\linewidth]{../ch3-napops/output/plots/Fig7-edr-change_small.png}
	\caption{\label{fig:ch1-edr-change}Change in effective detection radius when moving from a roadside survey to an offroad survey (i.e., $\Delta EDR = EDR_{Roadside} - EDR_{Offroad}$) against varying values of forest coverage, for the four top families (by sample size) modelled in NA-POPS. Red dashed line indicates a change in EDR of 0. Grey lines are species-specific changes in EDRs within each family, and black lines are mean family changes in EDR. Lines below the red dashed line indicate that the EDR for roadside surveys is less than the EDR for an offroad survey for that species, and lines above the red dashed line indicate that the EDR for roadside surveys is greater than the EDR for an offroad survey for that species.}
\end{figure}

\subsection{Case Study: American Robin \textit{Turdis migratorius}}

\par For removal modelling, the most parsimonious model for American Robin ($n_{removal} = 72620$) was Model 9, which contained linear and quadratic terms for TSSR and OD (Table 1 TO DO). In the current version of the NA-POPS database, there are at least some removal data from much of this species’ range, but also a strong spatial bias where most data are from the west and relatively few data cover the core of the species’ range in the East (\autoref{fig:ch1-removal-species}). The removal data for American Robin were much more balanced across the relevant parameter space, covering the standard seasons and times of day for point counts. The species’ relative availability is at a peak in late June (approximately OD 180) and 0.5 hours before sunrise, and its absolute availability is very high (almost 1.0) within a 10-minute point count conducted at the peak time of day and season.

\begin{figure}[h]
	\centering
	\includegraphics[width=\linewidth]{../ch3-napops/output/plots/Fig8-removal-species.png}
	\caption{\label{fig:ch1-removal-species}Summary of removal modelling for American Robin (Turdus migratorius, n = 72,620), including (A) spatial coverage of removal sampling, (B) covariate space for Ordinal Day and Time Since Sunrise, (C) predicted probability of availability against Ordinal Day for surveys of 1, 3, 5, and 10 minutes in duration, and (D) predicted probability of availability against Time Since Sunrise for surveys of 1, 3, 5, and 10 minutes in duration.}
\end{figure}

\par For distance modelling, American Robin ($n_{distance} = 98775$) had the full model of roadside effect, forest coverage effect, and an interaction term (Model 5) selected as most parsimonious by AIC (Table 2 TO DO). There are some distance data for this species across the southern and eastern portion of the species’ range, and also a strong western bias similar to the removal data (\autoref{fig:ch1-distance-species}). The current version of the NA-POPS database includes distance data for on- and off-road, and in both forested and non-forested landscapes, but there is also some bias in the parameter space with more data from forested landscapes and from off-road survey sites. Perceptibility of American Robin is generally greater in non-forested sites than in forested sites, and greater off-road than on-road.  

\begin{figure}[h]
	\centering
	\includegraphics[width=\linewidth]{../ch3-napops/output/plots/Fig9-distance-species.png}
	\caption{\label{fig:ch1-distance-species}Summary of distance modelling for American Robin (Turdus migratorius, n = 98775), including (A) spatial coverage of distance sampling, (B) covariate space for Forest Coverage and Roadside status, and predicted perceptibility against distance from observer for off-road and on-road surveys in (C) forested environments and (D) non-forested environments. }
\end{figure}

\section{Discussion}

\par NA-POPS: Point count Offsets for Population Sizes of North American Landbirds is a collaborative project that has generated empirical estimates of detectability in a range of common observation conditions for 338 species of landbirds in North America. This accounts for 75.4\% of the 448 species of landbirds considered in Partners in Flight’s 2016 Landbird Conservation Plan \citep{rosenberg_partners_2016}. This monumental effort fills a well-known gap in the literature, in that we have previously lacked a systematic way to generate detection estimates across species \citep{stanton_estimating_2019}. Past efforts by BAM \citep{matsuoka_using_2012, solymos_calibrating_2013, solymos_evaluating_2018} have started to target this, and NA-POPS serves as an extension of their work. The NA-POPS estimates of detectability can be used to integrate observations among diverse survey protocols (variations in duration and distance) and under varying survey conditions (forest-cover, roadside vs off-road, time of day, etc.), including BBS counts, eBird stationary counts, and IMBCR point-counts. They can also be used to inform detectability corrections in individual studies, as offsets or informative priors for detectability estimation. Most directly, these offsets provide an analytically coherent and empirical approach to improving estimates of population sizes of North American landbirds. 

\par NA-POPS has relatively good overall coverage in the west due to data contributed from the Avian Knowledge Network, and the boreal region with data contributed from the Boreal Avian Modelling project. By contrast, there are gaps in the NA-POPS coverage in the south-central and south-eastern portions of the continent. This could be supplemented with state/province-specific Breeding Bird Atlas point counts, which are key targets for future data to be added to NA-POPS. Additionally, NA-POPS requires better coverage of the large number of species that occur in Mexico and the south-western United States \citep{ruiz_gutierrez_proalas_2020}. Finally, with the increase in creation and use of bioacoustic data from tools such as autonomous recording units (ARUs), localization techniques \citep{hedley_direction--arrival_2017} and sound pressure level curves \citep{yip_sound_2020} could be used to estimate distances to singing birds from sound recordings, which could allow for a plethora of additional data to be added in to the NA-POPS database to inform detectability estimates. These gaps mentioned here mean that the species modelled in NA-POPS likely have some taxonomic bias to it, in that species that occur in the southwest of the United States (e.g., desert specialities, species who range occurs more in Mexico than the US), or species that have low population sizes to begin with (e.g., Kirtland’s Warbler \textit{Setophaga kirtlandii}, Bicknell’s Thrush \textit{Catharus bicknelli}) will be systematically underrepresented. By making use of targeted data mentioned here, we can attempt to fill in these taxonomic biases. 

\par The infrastructure of NA-POPS on the GitHub Organization has been set up for quick and simple continual integration of data sets, whether they are brand new to the analyses or updates of previous data sets. For any incoming data set, a new repository is created, and all steps in the Methods section of this paper are followed to standardize these data, add them to the data set to be modelled, and create a new set of covariates for each species given these new data. Updates to previous data sets can be done using the same approach, except a new project repository does not have to be created. This ease of incorporating new data means that the geographic, temporal, and species coverage of NA-POPS can continue to improve, allowing for estimates of coefficients to be refined as new data are integrated.

\subsection{Removal Modelling and Estimation of Cue Rate}

\par In many cases, the NA-POPS results of model selection among removal models and cue rate estimation align with previous studies using comparable methods \citep{solymos_calibrating_2013, solymos_evaluating_2018}. For the removal models, we used TSSR, OD, and their quadratic terms as covariates, because it is well known that bird availability is affected by both of these variables \citep{wilson_reliability_1985}. As expected, most species (88\%) had at least one of these covariates in their AIC-selected best model, and the best model for approximately half of species included at least one quadratic term. 

\par We found that 50.2\% of species included non-linear effects of TSSR on availability in their best model, somewhat fewer than the 70\% reported by \citet{solymos_evaluating_2018}. This could be due to NA-POPS modelling more species and therefore picking up more species that had only linear responses to TSSR. It could also be due to the range of the covariates of TSSR, in that many point count surveys tend to begin just before, or even at, the peak of dawn chorus, and so in species where most data are from the peak of dawn chorus or later, there may be insufficient data to support a curved relationship.

\par We found much greater support for non-linear effects of OD on availability than \citet{solymos_evaluating_2018}, which may be partly due to variation in the included species and also a result of an improved estimation approach. Whereas \citet{solymos_evaluating_2018} found that only 29\% of modelled species had nonlinear responses to OD, that was the case for 62\% of species modelled by NA-POPS. Although this could also be due to modelling more species, it is likely because we standardized our OD variables slightly differently than previous studies, in that we both scaled and centered our OD variables (where previous studies only scaled them). This additional centering of the variables ensured that models with and OD and OD2 term do not suffer from collinearity of the two variables, because squaring strictly positive terms will result in terms that are also strictly positive and therefore will be highly correlated. Thus, our models that included an OD and OD2 term likely performed better than previous analyses that did not centre the variables \citep{solymos_calibrating_2013, solymos_evaluating_2018}. 

\subsection{Distance Modelling and Estimation of Effective Detection Radius}

\par For the distance models, we used forest coverage and roadside status as covariates. The effect of forest coverage was expected to account for the attenuation of sound and light through forested vs. non-forested environments; that is, we would expect to be able to hear and/or see the same species at a further distance in a non-forested environment vs. a forested environment \citep{yip_sound_2017}. The effect of roadside status was expected to account for lesser sound attenuation on a road compared to an off-road environment \citep{yip_sound_2017}, but to also potentially account for the ability of an observer to perceive bird sounds when near a potentially loud road, compared to an off-road environment \citep{pacifici_effects_2008, cooke_road_2020}. The roadside status of a survey could also influence the ability of an observer to visually detect birds, with potentially higher detectability during roadside surveys where views are less obstructed compared to intact habitats. We also examined the effect of the interaction between these two variables. Given these previous studies of the effects of open/closed environments and roads on sound attenuation, and the results of \citet{solymos_calibrating_2013} when considering tree cover, we also expected that most species modelled in NA-POPS would have at least one of these covariates in their selected best model as chosen by AIC. Indeed, this was the case as only 7.5\% of species had the null model chosen.

\par When considering forest coverage as a covariate, our results were similar to those found in previous studies \citep{solymos_calibrating_2013, yip_sound_2017}, in that there was a small but non-zero effect of forest coverage on effective detection radius. That is, as forest coverage increases, the effective detection radius of a bird tends to decrease. This aligns with previous studies that have found that sound tends to attenuate quicker in forested environments than non-forested environments \citep{yip_sound_2017}.

\par Interestingly, the effective detection radius was greater in offroad surveys compared to roadside surveys for many species, contrary to earlier studies that have shown greater detectability on roadsides \citep{yip_sound_2017}. This was true for the American Robin example (\autoref{fig:ch1-distance-species}): for both forested and non-forested environments, the perceptibility of American Robin was greater in offroad surveys than roadside surveys. However, this pattern varied greatly by species, and appeared to interact with forest coverage (\autoref{fig:ch1-edr-change}). Although \citet{yip_sound_2017} showed that effective detection radius (and therefore perceptibility) is increased with roadside surveys due to decreased sound attenuation from the road surface, \citet{cooke_road_2020} showed that bird detectability with a selection of European birds was negatively associated with roadside surveys, particularly roadsides with heavier traffic. With this in mind, effects of roadside surveys compared to offroad surveys may be more difficult to pick up without accounting for traffic, and future versions of NA-POPS should consider differences in road types (e.g., major arterial road vs minor road), or an estimate of traffic.

\subsection{Selection of Null Removal or Distance Models}

\par Models that include covariates were favoured over the null models for most species (\autoref{fig:ch1-heatmap}). However, even for some of the species where the null model was best, the covariate models are likely still useful. For example, the null removal model was selected for MacGillivray’s Warbler \textit{Geothlypis tolmiei} and the null distance model was selected for Vesper Sparrow \textit{Pooecetes gramineus}, despite the fact that both species had a large number of detections (10,120 and 19,760, respectively). In both cases, the differences in AIC between the top selected null model and the more complex models were very close: for MacGillivray’s Warbler, the four next best models were models 4, 2, 6, and 5 with $\Delta$AICs of 1.01, 1.33, 2.33, and 2.96, respectively; for Vesper Sparrow, the four next best models (i.e., the remaining distance models) were models 3, 2, 4, and 5 with $\Delta$AICs of 1.59, 2.00, 3.58, and 5.45, respectively. Essentially, the models could be considered “tied”. This pattern holds for several species with null models selected as their best model, but with large sample sizes. 

\par While it is certainly possible that these species’ detectabilities are not in fact affected by the covariates used for these models, researchers wanting to apply these detectability offsets to their own point count data sets could consider using the more complex models of detectability if they have the ancillary data in their own point counts. In the \ttt{napops} R package, while we have an explicit “best model” argument for all the species when choosing covariates, we also allow for other covariates to be chosen if the researcher feels those covariates are relevant. For example, a practitioner that is wanting to generate densities for a species where the “null” model is chosen as the best model for explaining EDR, could still consider using the "road" model of EDR if they have the roadside status of each survey available to them; the BBS data would always be a roadside status survey, so the researcher could simply opt to use the roadside EDR for that roadside survey. On the other hand, if a practitioner feels that a model with forest coverage as a variable or time since sunrise as a variable is more useful, then they will have access to those models and should use those models. Finally, if the practitioner has none of these ancillary data for a given survey, then they could still simply use the null model for a species, even if a full model was chosen. We note that generally the "best" model is a compromise given the data, which is why “best” model could change when we have more data. Focused research (i.e., more data collected) allows for these shifts, and in the meantime, one can pick a different model if its biological mechanism is favoured. 

\subsection{Model Improvements}
\par Given the large number of observations compiled by NA-POPS to date, and the potential for many more data sets to be added, the NA-POPS project is in an excellent position to use more sophisticated modelling techniques that take full advantage of all the data we have. For example, we could consider more specific covariates to be used in both the removal and distance modelling aspects. For removal modelling, we could consider adding an effect that accounts for differences in timing of breeding and territorial defence across the continent. For example, in a widespread species such as American Robin, local spring times in the western part of the continent means that peak singing day could be significantly earlier in those areas compared to eastern North America where local spring happens much later. One further suggestion would be to use a landcover variable that captures local plant emergence times, similar to the “Local Spring” variable used in \citet{solymos_evaluating_2018}. Alternatively, a more complex version would be to include latitude and longitude spline functions in a Generalized Additive Model (GAM), which could facilitate the sharing of information across BCRs \citep{crainiceanu_bayesian_2005, wood_generalized_2017, pedersen_hierarchical_2019}. For distance modelling, a random observer effect could be incorporated into the model to account for differences in observer abilities. This could be particularly important if different projects train observers to detect birds differently (e.g., exact distances with rangefinders vs estimating binned distances), but could also be important to account for potential differences in hearing ability for particularly high-pitched birds such as Blackpoll Warblers. We note that although there is a plethora of additional variables we could consider, one of the main goals of NA-POPS is to make these estimates available and usable to the end-user, and so we must balance modelling with potentially relevant variables and modelling with variables that the end user will be able to access for their point-count data.

\par We can also consider ways to share information across multiple species. For example, we could share information phylogenetically by taking advantage of patterns of detectability within families (e.g., \autoref{fig:ch1-removal-family}, \autoref{fig:ch1-distance-family}, \autoref{fig:ch1-edr-change}), or by sharing information among species based on traits that are known to account for some differences in detectability \citep{johnston_species_2014, solymos_phylogeny_2018}. This could be done by developing a multi-species modelling framework that allows the sharing of information between similar species. That is, data-deficient species could borrow information from data-rich species that share similar phylogeny/traits. 

\par A flexible way to share information between units is through the use of hierarchical Bayesian modelling. These models have the additional benefit of allowing informative priors when we have existing information for a particular species (e.g., singing rate or effective detection radius). Bayesian models with informative priors would allow us to incorporate expert opinion into the analysis, such as the “Detection distance adjustment” used in the Partners in Flight Population Estimation Database, which are similar to our empirically derived EDRs \citep{rosenberg_partners_2016, will_handbook_2020}. Additionally, in developing a hierarchical Bayesian framework, data-poor BCRs or data-poor species could have a well-chosen, informative prior to fall back on to improve estimates. Previous studies have had success with using hierarchical Bayesian modelling for estimating detectability \citep{amundson_hierarchical_2014, sollmann_hierarchical_2016}, and so a Bayesian implementation of the QPAD methodology could serve as a useful addition to these previous successes by taking advantage of QPAD’s ability to analyze heterogeneous data.

\section{Summary}

\par NA-POPS is the first open-access database of detectability functions for over 300 species of North American landbirds. Our goal is to continue to grow the database to include more species and broaden the spatial coverage, and to further refine the models. The detectability functions generated from NA-POPS can be used to translate bird abundance into estimates of true density, and can play a crucial role in integrating disparate data sources into an integrated modelling framework. Additionally, systematic estimates of effective detection radius produced by the distance modelling component of NA-POPS, using covariates of roadside status and forest coverage, can be used to improve population estimates of North American landbirds, by accounting for detection biases in roadside surveys such as the BBS. NA-POPS is already a collaborative project, involving several agencies from across North America, but more partners are required to address spatial gaps and facilitate improved modelling. We invite researchers with bird point count data that use a removal or distance sampling approach to contribute to the further growth of the NA-POPS database.