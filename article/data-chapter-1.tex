\chapter{Point count offsets for estimating population sizes of North American landbirds}

\section{Abstract}
Bird monitoring in North America over several decades has generated many open databases, housing millions of structured and semi-structured bird observations. These provide the opportunity to estimate bird densities and population sizes, once variation in factors such as underlying field methods, timing, land cover, proximity to roads, and uneven spatial coverage are accounted for. To facilitate that integration across databases, we introduce NA-POPS: Point Count Offsets for Population Sizes of North American Landbirds. NA-POPS is a large-scale, multi-agency project providing an open-source database of detectability functions for all North American landbirds. These detectability functions allow the integration of data from across disparate survey methods using the QPAD approach, which considers the probability of detection (q) and availability (p) of birds in relation to area (a) and density (d). To date, NA-POPS has compiled over 7.1 million data points spanning 292 projects from across North America, and produced detectability functions for 338 landbird species. Here, we describe the methods used to curate these data and generate these detectability functions, as well as the open-access nature of the resulting database. 

\section{Introduction}
\par The broad scale monitoring of birds in North America over the past several decades has resulted in the availability of millions of bird observations in open databases that span most of the continent. Individual programs such as the North American Breeding Bird Survey (BBS; \citep{hudson_role_2017}, Sauer et al. 2017), the Integrated Monitoring in Bird Conservation Regions (IMBCR; Pavlacky et al. 2017), the Boreal Avian Modelling Project (BAM; Cumming et al. 2010), and eBird (Sullivan et al. 2014) provide a great deal of information on relative abundance over time and space. Partners in Flight (PIF) has previously estimated population sizes of landbirds using BBS data and a series of sophisticated expert-informed equations to extrapolate survey-level counts to total abundance within defined geographic regions (Rosenberg \& Blancher 2005; Will et al. 2020). These population sizes have been used to inform reports such as the 2019 State of Canada’s Birds report (North American Bird Conservation Initiative Canada 2019), and to show the loss of nearly 3 billion North American birds since the 1970s (Rosenberg et al. 2019).

\subsection{Test}
hello

\lipsum[3-4]

\begin{figure}[h]
	\centering
	\includegraphics[width=\linewidth]{coherence_k500}
	\caption{\label{fig:coherence}This caption explains the above plot.}
\end{figure}

\lipsum[5]
Looking at the figure above (\autoref{fig:coherence}), I don't remember what I was going to say.


\section{Second section}

\lipsum[6-7]

\section{Third section}

Everything I do is in R 4.0\citep{R-4.0.0}. \ttt{purrr}\citep{purrr} is a great package. I also used a
bunch of other packages\citep{qs,dplyr,ggplot2}.
